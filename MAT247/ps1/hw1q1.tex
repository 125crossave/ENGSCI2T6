% ***********************************************************
% ******************* PHYSICS HEADER ************************
% ***********************************************************
% Version 2
\documentclass[12pt]{article}
\usepackage{amsmath} % AMS Math Package
\usepackage{amsthm} % Theorem Formatting
\usepackage{amssymb}    % Math symbols such as \mathbb
\usepackage{graphicx} % Allows for eps images
\usepackage[dvips,letterpaper,margin=1in,bottom=0.7in]{geometry}
\usepackage{tensor}
 % Sets margins and page size
\usepackage{amsmath}

\renewcommand{\labelenumi}{(\alph{enumi})} % Use letters for enumerate
% \DeclareMathOperator{\Sample}{Sample}
\let\vaccent=\v % rename builtin command \v{} to \vaccent{}
\usepackage{enumerate}
\renewcommand{\v}[1]{\ensuremath{\mathbf{#1}}} % for vectors
\newcommand{\gv}[1]{\ensuremath{\mbox{\boldmath$ #1 $}}}
% for vectors of Greek letters
\newcommand{\uv}[1]{\ensuremath{\mathbf{\hat{#1}}}} % for unit vector
\newcommand{\abs}[1]{\left| #1 \right|} % for absolute value
\newcommand{\avg}[1]{\left< #1 \right>} % for average
\let\underdot=\d % rename builtin command \d{} to \underdot{}
\renewcommand{\d}[2]{\frac{d #1}{d #2}} % for derivatives
\newcommand{\dd}[2]{\frac{d^2 #1}{d #2^2}} % for double derivatives
\newcommand{\pd}[2]{\frac{\partial #1}{\partial #2}}
% for partial derivatives
\newcommand{\pdd}[2]{\frac{\partial^2 #1}{\partial #2^2}}
% for double partial derivatives
\newcommand{\pdc}[3]{\left( \frac{\partial #1}{\partial #2}
 \right)_{#3}} % for thermodynamic partial derivatives
\newcommand{\ket}[1]{\left| #1 \right>} % for Dirac bras
\newcommand{\bra}[1]{\left< #1 \right|} % for Dirac kets
\newcommand{\braket}[2]{\left< #1 \vphantom{#2} \right|
 \left. #2 \vphantom{#1} \right>} % for Dirac brackets
\newcommand{\matrixel}[3]{\left< #1 \vphantom{#2#3} \right|
 #2 \left| #3 \vphantom{#1#2} \right>} % for Dirac matrix elements
\newcommand{\grad}[1]{\gv{\nabla} #1} % for gradient
\let\divsymb=\div % rename builtin command \div to \divsymb
\renewcommand{\div}[1]{\gv{\nabla} \cdot \v{#1}} % for divergence
\newcommand{\curl}[1]{\gv{\nabla} \times \v{#1}} % for curl
\let\baraccent=\= % rename builtin command \= to \baraccent
\renewcommand{\=}[1]{\stackrel{#1}{=}} % for putting numbers above =
\providecommand{\wave}[1]{\v{\tilde{#1}}}
\providecommand{\fr}{\frac}
\providecommand{\RR}{\mathbb{R}}
\providecommand{\NN}{\mathbb{N}}
\providecommand{\seq}{\subseteq}
\providecommand{\e}{\epsilon}

\newtheorem{prop}{Proposition}
\newtheorem{thm}{Theorem}[section]
\newtheorem{axiom}{Axiom}[section]
\newtheorem{p}{Problem}[section]
\usepackage{cancel}
\newtheorem*{lem}{Lemma}
\theoremstyle{definition}
\newtheorem*{dfn}{Definition}
 \newenvironment{s}{%\small%
        \begin{trivlist} \item \textbf{Solution}. }{%
            \hspace*{\fill} $\blacksquare$\end{trivlist}}%
% ***********************************************************
% ********************** END HEADER *************************
% ***********************************************************

\begin{document}
\section*{Question 1}\LARGE\pagestyle{empty}
a) Letting $A = \begin{pmatrix}
    a &b\\
    c & d
\end{pmatrix}$, where $a,b,c,d \in \textbf{R}$, we have

$$Av_1 = \lambda_1v_1 \implies \begin{pmatrix}
    a & b\\
    c & d
\end{pmatrix}\begin{pmatrix}
    1\\
    -1
\end{pmatrix}=\begin{pmatrix}
    1\\
    -1
\end{pmatrix}$$$$Av_2 = \lambda_2v_2 \implies \begin{pmatrix}
    a & b\\
    c & d
\end{pmatrix}\begin{pmatrix}
    -2\\
    1
\end{pmatrix}=-2\begin{pmatrix}
    -2\\
    1
\end{pmatrix}$$
which yields the systems of equations $a - b = 1, -2a + b = 4$ and $c - d = -1, -2c + d = -2$. Solving these we obtain $a = -5, b = -6, c = 3, d = 4$ so that $$A = \begin{pmatrix}
    -5 &-6\\
    3 & 4
\end{pmatrix}$$

\noindent b) Suppose $A$ is not unique, i.e. that $A_i, A_j$ $\in M_{2\times2}(\textbf{R})$ exist s.t. $A_i \neq A_j$ and $$A_iv_1 = \lambda_1v_1, A_jv_1 = \lambda_1v_1, A_iv_2 = \lambda_2v_2, A_jv_2 = \lambda_2v_2$$Then $(A_i - A_j)v_1 = (\lambda_1 - \lambda_1)v_1 = (A_i - A_j)v_2 = (\lambda_2 - \lambda_2)v_2 = 0$, so $v_1, v_2 \in$ null$(A_i - A_j)$. But, since $v_1$ and $v_2$ are linearly independent vectors, this means dim null$(A_i - A_j) = 2$, i.e. rank$(A_i - A_j)$ = 0 by the rank-nullity theorem (since dim $\textbf{R}^2$ = dim null$(A_i - A_j)$ + rank$(A_i - A_j)$). This is only possible if $A_i - A_j = O$, i.e. $A_i = A_j$. Thus a contradiction arises from our assumption that $A_i \neq A_j$ and it follows that $A$ must be unique.
\newpage
\noindent c) Plugging in $A$, we find that $$A^3 - A^2 + 3I = \begin{pmatrix}
    -21 &-24\\
    12 & 15
\end{pmatrix}$$Denote this matrix \textit{Q}. The eigenvalues are the roots of $det(Q - \lambda I)$: $$0 = (-21 - \lambda)(15 - \lambda) - (-24)(12)\implies \lambda = 3, -9$$Eigenvectors of $Q$ are $v\neq 0$ such that $Qv = 3v$ or $Qv = -9v$. Let $v$ satisfying the former case be $v_1$ and $v$ satisfying the latter be $v_2$. Then $v_1\in$ null$(Q-3I)$ and $v_2 \in$ null$(Q + 9I)$. Letting $v_1 = \begin{pmatrix}
    a\\
    b
\end{pmatrix}$, $v_2 = \begin{pmatrix}
    c\\
    d
\end{pmatrix}$, we find that the systems of equations $-24a - 24b = 0$, $12a + 12b = 0$ and $-12c - 24d = 0$, $12c + 24d = 0$ must be satisfied. Then $a = -b$ and $c = -2d$, so taking $b$ and $d$ respectively as the free variables we have $$v_1 = \begin{pmatrix}
    -1\\
    1
\end{pmatrix}, v_2 = \begin{pmatrix}
    -2\\
    1
\end{pmatrix}$$
\end{document}