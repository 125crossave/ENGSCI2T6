% ***********************************************************
% ******************* PHYSICS HEADER ************************
% ***********************************************************
% Version 2
\documentclass[12pt]{article}
\usepackage{amsmath} % AMS Math Package
\usepackage{amsthm} % Theorem Formatting
\usepackage{amssymb}    % Math symbols such as \mathbb
\usepackage{graphicx} % Allows for eps images
\usepackage[dvips,letterpaper,margin=1in,bottom=0.7in]{geometry}
\usepackage{tensor}
 % Sets margins and page size
\usepackage{amsmath}

\renewcommand{\labelenumi}{(\alph{enumi})} % Use letters for enumerate
% \DeclareMathOperator{\Sample}{Sample}
\let\vaccent=\v % rename builtin command \v{} to \vaccent{}
\usepackage{enumerate}
\renewcommand{\v}[1]{\ensuremath{\mathbf{#1}}} % for vectors
\newcommand{\gv}[1]{\ensuremath{\mbox{\boldmath$ #1 $}}}
% for vectors of Greek letters
\newcommand{\uv}[1]{\ensuremath{\mathbf{\hat{#1}}}} % for unit vector
\newcommand{\abs}[1]{\left| #1 \right|} % for absolute value
\newcommand{\avg}[1]{\left< #1 \right>} % for average
\let\underdot=\d % rename builtin command \d{} to \underdot{}
\renewcommand{\d}[2]{\frac{d #1}{d #2}} % for derivatives
\newcommand{\dd}[2]{\frac{d^2 #1}{d #2^2}} % for double derivatives
\newcommand{\pd}[2]{\frac{\partial #1}{\partial #2}}
% for partial derivatives
\newcommand{\pdd}[2]{\frac{\partial^2 #1}{\partial #2^2}}
% for double partial derivatives
\newcommand{\pdc}[3]{\left( \frac{\partial #1}{\partial #2}
 \right)_{#3}} % for thermodynamic partial derivatives
\newcommand{\ket}[1]{\left| #1 \right>} % for Dirac bras
\newcommand{\bra}[1]{\left< #1 \right|} % for Dirac kets
\newcommand{\braket}[2]{\left< #1 \vphantom{#2} \right|
 \left. #2 \vphantom{#1} \right>} % for Dirac brackets
\newcommand{\matrixel}[3]{\left< #1 \vphantom{#2#3} \right|
 #2 \left| #3 \vphantom{#1#2} \right>} % for Dirac matrix elements
\newcommand{\grad}[1]{\gv{\nabla} #1} % for gradient
\let\divsymb=\div % rename builtin command \div to \divsymb
\renewcommand{\div}[1]{\gv{\nabla} \cdot \v{#1}} % for divergence
\newcommand{\curl}[1]{\gv{\nabla} \times \v{#1}} % for curl
\let\baraccent=\= % rename builtin command \= to \baraccent
\renewcommand{\=}[1]{\stackrel{#1}{=}} % for putting numbers above =
\providecommand{\wave}[1]{\v{\tilde{#1}}}
\providecommand{\fr}{\frac}
\providecommand{\RR}{\mathbb{R}}
\providecommand{\NN}{\mathbb{N}}
\providecommand{\seq}{\subseteq}
\providecommand{\e}{\epsilon}

\newtheorem{prop}{Proposition}
\newtheorem{thm}{Theorem}[section]
\newtheorem{axiom}{Axiom}[section]
\newtheorem{p}{Problem}[section]
\usepackage{cancel}
\newtheorem*{lem}{Lemma}
\theoremstyle{definition}
\newtheorem*{dfn}{Definition}
 \newenvironment{s}{%\small%
        \begin{trivlist} \item \textbf{Solution}. }{%
            \hspace*{\fill} $\blacksquare$\end{trivlist}}%
% ***********************************************************
% ********************** END HEADER *************************
% ***********************************************************

\begin{document}\pagestyle{empty}
\section*{Question 3}\LARGE
Since $$A - \lambda I = \left(\begin{array}{cccccc}
-\lambda&1&0&\cdots&\cdots&0\\
0&-\lambda&1&\cdots&\cdots&0\\
\cdots&\cdots&\cdots&\cdots&\cdots&\cdots\\
\cdots&\cdots&\cdots&\cdots&\cdots&\cdots\\
0&0&0&\cdots&\cdots&1\\
1&0&0&\cdots&\cdots&-\lambda\\
\end{array}\right)$$it follows that $p_A(\lambda) = det(A-\lambda I) = (-1)^{1+1}(-\lambda)(-\lambda)^{n-1} + (-1)^{n+1}(1)1^{n-1}$ through Laplace expansion of the determinant along the first column (and using the fact that the determinant of triangular matrices is the product of the diagonal entries). If $n$ odd, then $n-1$ and $n+1$ are even, so$$p_A(\lambda) = (-\lambda)\lambda^{n-1} + 1 = 1 - \lambda^n$$ and the roots of $p_A(\lambda)$ (i.e. the eigenvalues of $A$) are the solutions to $\lambda^n = 1$. Likewise, if $n$ even then $n-1$ and $n+1$ are odd, so $$p_A(\lambda) = (-\lambda)(-\lambda)^{n-1} - 1 = \lambda^n - 1$$ and the eigenvalues of $A$ are the solutions to $\lambda^n = 1$. Thus we conclude that the eigenvalues of $A$ are $n$th roots of unity, i.e. solve $$\lambda^n = 1$$ when $A \in M_{n\times n}(\textbf{C})$.
\newpage
\noindent For eigenvectors, first note that left-multiplying $A$ with any column vector $x = \begin{pmatrix}
    x_0 \\ x_1\\ ...\\ x_{n-1}
\end{pmatrix}$ produces a vector wherein all entries of $x$ are shifted upward by 1 row (with the top entry being cycled down to the bottom):
$$Ax = \begin{pmatrix}
    x_{1} \\ x_2\\ ...\\ x_{0}
\end{pmatrix}$$Since $\lambda^n = 1$, we observe that $v = \begin{pmatrix}
    \lambda^0 \\ \lambda^1\\ ...\\ \lambda^{n-1}
\end{pmatrix}$ satisfies $Av = \lambda v$ and is thus an eigenvector. Since the $n$ roots of $\lambda^n = 1$ are $e^{\frac{2\omega\pi i}{n}}$, where $\omega = 0, 1, ..., n-1$, we conclude that the eigenvectors of $v$ are $v_{\omega}$ s.t.$$v_{\omega} = \begin{pmatrix}
    (e^{\frac{2\omega\pi i}{n}})^0 \\ (e^{\frac{2\omega\pi i}{n}})^1\\ ...\\(e^{\frac{2\omega\pi i}{n}})^{n-1}
\end{pmatrix}$$ where $\omega = 0, 1, ..., n-1$.
\end{document}