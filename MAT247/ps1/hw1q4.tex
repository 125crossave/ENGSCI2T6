% ***********************************************************
% ******************* PHYSICS HEADER ************************
% ***********************************************************
% Version 2
\documentclass[12pt]{article}
\usepackage{amsmath} % AMS Math Package
\usepackage{amsthm} % Theorem Formatting
\usepackage{amssymb}    % Math symbols such as \mathbb
\usepackage{graphicx} % Allows for eps images
\usepackage[dvips,letterpaper,margin=1in,bottom=0.7in]{geometry}
\usepackage{tensor}
 % Sets margins and page size
\usepackage{amsmath}

\renewcommand{\labelenumi}{(\alph{enumi})} % Use letters for enumerate
% \DeclareMathOperator{\Sample}{Sample}
\let\vaccent=\v % rename builtin command \v{} to \vaccent{}
\usepackage{enumerate}
\renewcommand{\v}[1]{\ensuremath{\mathbf{#1}}} % for vectors
\newcommand{\gv}[1]{\ensuremath{\mbox{\boldmath$ #1 $}}}
% for vectors of Greek letters
\newcommand{\uv}[1]{\ensuremath{\mathbf{\hat{#1}}}} % for unit vector
\newcommand{\abs}[1]{\left| #1 \right|} % for absolute value
\newcommand{\avg}[1]{\left< #1 \right>} % for average
\let\underdot=\d % rename builtin command \d{} to \underdot{}
\renewcommand{\d}[2]{\frac{d #1}{d #2}} % for derivatives
\newcommand{\dd}[2]{\frac{d^2 #1}{d #2^2}} % for double derivatives
\newcommand{\pd}[2]{\frac{\partial #1}{\partial #2}}
% for partial derivatives
\newcommand{\pdd}[2]{\frac{\partial^2 #1}{\partial #2^2}}
% for double partial derivatives
\newcommand{\pdc}[3]{\left( \frac{\partial #1}{\partial #2}
 \right)_{#3}} % for thermodynamic partial derivatives
\newcommand{\ket}[1]{\left| #1 \right>} % for Dirac bras
\newcommand{\bra}[1]{\left< #1 \right|} % for Dirac kets
\newcommand{\braket}[2]{\left< #1 \vphantom{#2} \right|
 \left. #2 \vphantom{#1} \right>} % for Dirac brackets
\newcommand{\matrixel}[3]{\left< #1 \vphantom{#2#3} \right|
 #2 \left| #3 \vphantom{#1#2} \right>} % for Dirac matrix elements
\newcommand{\grad}[1]{\gv{\nabla} #1} % for gradient
\let\divsymb=\div % rename builtin command \div to \divsymb
\renewcommand{\div}[1]{\gv{\nabla} \cdot \v{#1}} % for divergence
\newcommand{\curl}[1]{\gv{\nabla} \times \v{#1}} % for curl
\let\baraccent=\= % rename builtin command \= to \baraccent
\renewcommand{\=}[1]{\stackrel{#1}{=}} % for putting numbers above =
\providecommand{\wave}[1]{\v{\tilde{#1}}}
\providecommand{\fr}{\frac}
\providecommand{\RR}{\mathbb{R}}
\providecommand{\NN}{\mathbb{N}}
\providecommand{\seq}{\subseteq}
\providecommand{\e}{\epsilon}

\newtheorem{prop}{Proposition}
\newtheorem{thm}{Theorem}[section]
\newtheorem{axiom}{Axiom}[section]
\newtheorem{p}{Problem}[section]
\usepackage{cancel}
\newtheorem*{lem}{Lemma}
\theoremstyle{definition}
\newtheorem*{dfn}{Definition}
 \newenvironment{s}{%\small%
        \begin{trivlist} \item \textbf{Solution}. }{%
            \hspace*{\fill} $\blacksquare$\end{trivlist}}%
% ***********************************************************
% ********************** END HEADER *************************
% ***********************************************************

\begin{document}\pagestyle{empty}
\section*{Question 4}\LARGE
a) The characteristic polynomial $p_A(\lambda)$ is $$det(A-\lambda I) = det\begin{pmatrix}
    3 - \lambda & -9 & 1\\
    2 & -8 - \lambda & -2\\
    3 & -3 & 1 - \lambda
\end{pmatrix}$$ which, using the Laplace expansion along the third column, becomes $p_A(\lambda) = (-1)^{3 + 1}(-6 - 3(-8-\lambda)) + (-1)^{3 + 2}(-2)((3-\lambda)(-3) + 27) + (-1)^{3+3}(1-\lambda)((3-\lambda)(-8-\lambda) + 18) = -(\lambda+6)(\lambda+2)(\lambda-4)$. The roots of this polynomial are $\lambda_1 = -6, \lambda_2 = -2, \lambda_3 = 4$.

b) To find the corresponding eigenvectors we must find vectors in null$(A - \lambda_i I)$ for each $i \in  \{1, 2, 3\}$. For $\lambda_1$, we have $$\text{null}(A - \lambda_1I) = \text{null}\begin{pmatrix}
    9 & -9 & 1\\
    2 & -2 & -2\\
    3 & -3 & 7
\end{pmatrix} = \text{null}\begin{pmatrix}
    1 & -1 & 0\\
    0 & 0 & 1\\
    0 & 0 & 0
    \end{pmatrix}$$ after performing row reduction; hence we obtain the system of equations $x_1 = x_2, x_3 = 0$. Taking $x_2$ as the free variable, we find that an eigenvector corresponding to $\lambda_1$ is $v_1 = \begin{pmatrix}
        1 \\ 1 \\ 0
    \end{pmatrix}$. For $\lambda_2$, we have $$\text{null}(A - \lambda_2I) = \text{null}\begin{pmatrix}
    5 & -9 & 1\\
    2 & -6 & -2\\
    3 & -3 & 3
\end{pmatrix} = \text{null}\begin{pmatrix}
    1 & 0 & 2\\
    0 & 1 & 1\\
    0 & 0 & 0
    \end{pmatrix}$$ after performing row reduction; hence we obtain the system of equations $x_1 = -2x_3$ and $x_2 = -x_3$. Taking $x_3$ as the free variable, we find that an eigenvector corresponding to $\lambda_2$ is $v_2 = \begin{pmatrix}
        -2 \\ -1 \\ 1
    \end{pmatrix}$. For $\lambda_3$, we have $$\text{null}(A - \lambda_3I) = \text{null}\begin{pmatrix}
    -1 & -9 & 1\\
    2 & -12 & -2\\
    3 & -3 & -3
\end{pmatrix} = \text{null}\begin{pmatrix}
    1 & 0 & -1\\
    0 & 1 & 0\\
    0 & 0 & 0
\end{pmatrix}$$ after performing row reduction, hence we obtain the system of equations $x_1 = x_3$, $x_2 = 0$. Taking $x_3$ as the free variable, we find that an eigenvector corresponding to $\lambda_3$ is $v_3 = \begin{pmatrix}
        1 \\ 0 \\ 1
    \end{pmatrix}$.

c) The columns of a diagonalizing matrix $C$ (where $C^{-1}AC = D$) are the eigenvectors of $A$ in the order (left-to-right) that the eigenvalues appear in the diagonal matrix $D$; hence $$C = \begin{pmatrix}
    1 & -2 & 1\\
    1 & -1 & 0\\
    0 & 1 & 1
\end{pmatrix}$$is the diagonalizing matrix corresponding to the diagonal matrix $$D = \begin{pmatrix}
    -6 & 0 & 0\\
    0 & -2 & 0\\
    0 & 0 & 4
\end{pmatrix}$$
\end{document}