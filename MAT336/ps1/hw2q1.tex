% ***********************************************************
% ******************* PHYSICS HEADER ************************
% ***********************************************************
% Version 2
\documentclass[12pt]{article}
\usepackage{amsmath} % AMS Math Package
\usepackage{amsthm} % Theorem Formatting
\usepackage{amssymb}    % Math symbols such as \mathbb
\usepackage{graphicx} % Allows for eps images
\usepackage[dvips,letterpaper,margin=1in,bottom=0.7in]{geometry}
\usepackage{tensor}
 % Sets margins and page size
\usepackage{amsmath}

\renewcommand{\labelenumi}{(\alph{enumi})} % Use letters for enumerate
% \DeclareMathOperator{\Sample}{Sample}
\let\vaccent=\v % rename builtin command \v{} to \vaccent{}
\usepackage{enumerate}
\renewcommand{\v}[1]{\ensuremath{\mathbf{#1}}} % for vectors
\newcommand{\gv}[1]{\ensuremath{\mbox{\boldmath$ #1 $}}}
% for vectors of Greek letters
\newcommand{\uv}[1]{\ensuremath{\mathbf{\hat{#1}}}} % for unit vector
\newcommand{\abs}[1]{\left| #1 \right|} % for absolute value
\newcommand{\avg}[1]{\left< #1 \right>} % for average
\let\underdot=\d % rename builtin command \d{} to \underdot{}
\renewcommand{\d}[2]{\frac{d #1}{d #2}} % for derivatives
\newcommand{\dd}[2]{\frac{d^2 #1}{d #2^2}} % for double derivatives
\newcommand{\pd}[2]{\frac{\partial #1}{\partial #2}}
% for partial derivatives
\newcommand{\pdd}[2]{\frac{\partial^2 #1}{\partial #2^2}}
% for double partial derivatives
\newcommand{\pdc}[3]{\left( \frac{\partial #1}{\partial #2}
 \right)_{#3}} % for thermodynamic partial derivatives
\newcommand{\ket}[1]{\left| #1 \right>} % for Dirac bras
\newcommand{\bra}[1]{\left< #1 \right|} % for Dirac kets
\newcommand{\braket}[2]{\left< #1 \vphantom{#2} \right|
 \left. #2 \vphantom{#1} \right>} % for Dirac brackets
\newcommand{\matrixel}[3]{\left< #1 \vphantom{#2#3} \right|
 #2 \left| #3 \vphantom{#1#2} \right>} % for Dirac matrix elements
\newcommand{\grad}[1]{\gv{\nabla} #1} % for gradient
\let\divsymb=\div % rename builtin command \div to \divsymb
\renewcommand{\div}[1]{\gv{\nabla} \cdot \v{#1}} % for divergence
\newcommand{\curl}[1]{\gv{\nabla} \times \v{#1}} % for curl
\let\baraccent=\= % rename builtin command \= to \baraccent
\renewcommand{\=}[1]{\stackrel{#1}{=}} % for putting numbers above =
\providecommand{\wave}[1]{\v{\tilde{#1}}}
\providecommand{\fr}{\frac}
\providecommand{\RR}{\mathbb{R}}
\providecommand{\NN}{\mathbb{N}}
\providecommand{\seq}{\subseteq}
\providecommand{\e}{\epsilon}

\newtheorem{prop}{Proposition}
\newtheorem{thm}{Theorem}[section]
\newtheorem{axiom}{Axiom}[section]
\newtheorem{p}{Problem}[section]
\usepackage{cancel}
\newtheorem*{lem}{Lemma}
\theoremstyle{definition}
\newtheorem*{dfn}{Definition}
 \newenvironment{s}{%\small%
        \begin{trivlist} \item \textbf{Solution}. }{%
            \hspace*{\fill} $\blacksquare$\end{trivlist}}%
% ***********************************************************
% ********************** END HEADER *************************
% ***********************************************************

\begin{document}\pagestyle{empty}
\section*{Question 1}\Large
\quad a) Notice that $$a_n = \frac{n!}{n^n}= \frac{(n)(n-1)(n-2)...(2)(1)}{(n)(n)(n)...(n)(n)}\leq \frac{1}{n}$$ due to each $n, n-1$, $n - 2$, ..., 1 in the numerator being smaller than or equal to the corresponding $n$ in the denominator. Since $0 \leq a_n \leq \frac{1}{n}\forall n\in \textbf{N}$, and $\lim_{n\to\infty}\frac{1}{n} = 0$ (result from lecture), we can apply the squeeze theorem: $$0 = \lim_{n\to\infty}0 \leq \lim_{n\to\infty}a_n \leq \lim_{n\to\infty}\frac{1}{n} = 0$$ Hence $\lim_{n\to\infty}a_n = 0$.

b) Multiply by 1 and simplify:
$$a_n = \frac{(\sqrt{n^2+4n-3} - n)(\sqrt{n^2+4n-3} + n)}{\sqrt{n^2+4n-3} + n} = \frac{n^2 + 4n - 3 - n^2}{\sqrt{n^2+4n-3} + n}$$
Multiply by 1 again and simplify further: $$a_n = \frac{(4n - 3)\frac{1}{n}}{(\sqrt{n^2+4n-3} + n)\frac{1}{n}} = \frac{4 - \frac{3}{n}}{\frac{\sqrt{n^2+4n-3}}{n} +1} = \frac{4 - \frac{3}{n}}{\sqrt{\frac{n^2+4n-3}{n^2}} +1}$$Using the algebraic limit laws of addition and multiplication, $$\lim_{n\to\infty}a_n = \lim_{n\to\infty}\frac{4 - \frac{1}{n}}{\sqrt{1 + \frac{4}{n} - \frac{3}{n^2}} +1} = \frac{\lim_{n\to\infty}(4 - \frac{3}{n})}{\lim_{n\to\infty}(\sqrt{1 + \frac{4}{n} - \frac{3}{n^2}} +1)}$$$$= \frac{\lim_{n\to\infty}(4) - 3\lim_{n\to\infty}\frac{1}{n}}{\sqrt{\lim_{n\to\infty}1 + 4\lim_{n\to\infty}\frac{1}{n} - 3(\lim_{n\to\infty}\frac{1}{n})(\lim_{n\to\infty}\frac{1}{n})} + \lim_{n\to\infty}1}$$and by the fact that $\lim_{n\to\infty}\frac{1}{n}=0$ (result from lecture), we have $$\lim_{n\to\infty}a_n = \frac{4 - 0}{1 + 1} = 2$$ Note that when we took the limit under the square root this does not follow directly from the algebraic limit laws, but nevertheless is a valid technique for taking limits of sequences in this context due to the fact that the quantity under the square root ($1 + \frac{4}{n} - \frac{3}{n^2})$ is always $\geq$ 0 for all n.
\end{document}